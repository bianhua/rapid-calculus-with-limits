\documentclass[10pt,a4paper]{book}
\usepackage[utf8]{inputenc}
\usepackage{amsmath}
\usepackage{amsfonts}
\usepackage{amssymb}
\usepackage{graphicx}
\usepackage{lmodern}
\usepackage{pgfplots}
\usepackage{multicol}
\usepackage{microtype}
\usepackage{tabularx}
\author{Bian Hua}
\title{Rapid Calculus with Limits}
\begin{document}
%Beginning business...
\maketitle
\tableofcontents

\chapter{Limits}
Modern calculus is mostly concerned with the concept of the \textit{limit}, as opposed to the
original approach, which was based on \textit{infinitesimals}. The differences between these
two approaches will be discussed more in the chapter on differentiation, but for now be content
in the knowledge that limits are useful.

Consider the function:
\[
	f(x)= {{x^2-1}\over{x-1}}
\]

With the graph:

\begin{tikzpicture}
  \begin{axis}[ 
    xlabel=$x$,
    ylabel={$f(x)= {{x^2-1}\over{x-1}}$},
    xmin=0, xmax=5,
    ymin=0, ymax=5,
  ] 
    \addplot[black] {(x^2-1)/(x-1)}; 
    \addplot[
    color=black,
    fill=white,
    mark=*,
    only marks
    ] coordinates{
    (1,2)
    };
  \end{axis}
\end{tikzpicture}

As the chart suggests, the function is not defined for $x=1$:
\[
f(1)= {\frac{1^2-1}{1-1}} = {\frac{0}{0}} = \text{undefined}
\]

Suppose, however, that we wanted to know what value a function \textit{approaches} as it nears
a point at which it may or may not be defined. Enter the limit.

Limits come in the form:
\[
\lim_{a \to b} f(a)
\]

Where $a$ is some variable and $b$ is some constant.

In the case of the function in the above example, we could find the value that $f(x)$ approaches
as $x$ approaches 1 by several methods.
\newline

\begin{tabularx}{\textwidth}{ l|X }
	\multicolumn{2}{c}{\textbf{Approaches to $\displaystyle \lim_{x \to 1} {\frac{x^2-1}{x-1}}$}} \\
	\hline
	\textbf{Visual} & A visual inspection of the graph shows that $f(1)\approx 2$, but this is far from rigorous. \\
	\textbf{Numerical} & Creating a table of values can demonstrate the value that the function approaches. For example, $f(0.99)=1.99$ and $f(1.01)=2.01$.\\
	\textbf{Algebraic} & By factoring out the expression in the denominator, we can create an equivalent function that does not have a hole at $x=1$:
	\[
	{\frac{x^2-1}{x-1}} = {\frac{(x-1)(x+1)}{x-1}} = x+1
	\]
	\\
	\textbf{L'H\^{o}pital's Rule} & As will be demonstrated in the chapter on derivative applications:
	\[
	\text{For }x=1, {\frac{x^2-1}{x-1}} = {\frac{\frac{d}{dx}\left(x^2-1\right)}{\frac{d}{dx}\left(x-1\right)}} = \frac{2x}{1} = 2x
	\]
	\\
\end{tabularx}

\section{Continuity}
The approaches described in the above table are only necessary if the function cannot be evaluated by means of simple \textbf{substitution}. As we will see in Section 1.3, however, the mere existence of the function at a point is not grounds for using substitution to evaluate the limit.

Strictly speaking, $\displaystyle \lim_{a \to b} f(a)$ has nothing at all to do with the value of $f(b)$. How, then, can we determine if $\displaystyle {\lim_{a \to b} f(a)} = f(b)$?

\textbf{Continuity} refers to the gradual progression of a function, regardless of scale. A function may be considered continuous on an interval if, regardless of how close we examine a section of the interval, it still appears to be a set of points that might be loosely termed "adjacent".

$\displaystyle {\lim_{a \to b} f(a)} = f(b)$ if two points $c$ and $d$ can be found such that $c < b < d$ and $f(a)$ is continuous on the interval $c < a < d$. That is, $f(a)$ is continuous about $a=b$.

\section{Existence}
The limit $\displaystyle \lim_{a \to b} f(a)$ exists only if $\displaystyle \lim_{a \to b^{+}} f(a) = \lim_{a \to b^{-}} f(a)$. That is, the function must approach the same value from either direction. The function does not have to equal the values of these limits for the limit to exist, as is demonstrated in Section 1.3.

As may be intuitive, oscillating functions such as $\sin x$ do not have limits as $x$ trends to infinity. Dampened oscillating functions, however, such as $\frac{\sin x}{x}$, do have such limits (in this case, 0). The case of the dampened oscillator is addressed by the \textbf{squeeze theorem}\footnote{The \textbf{squeeze theorem} states (among other things) that $\displaystyle \lim_{x \to \infty} {\frac{\sin x}{x}} = 0$, despite the continued oscillation.}.
\subsubsection{Problem Set 1}
Evaluate the limit.
\begin{multicols}{2}
\begin{enumerate}
	\item $\displaystyle \lim_{x \to 3} {x+4}$
    \item $\displaystyle \lim_{x \to 0} {\frac{1}{x}}$
    \item $\displaystyle \lim_{x \to \infty} {\frac{2-x}{x^{2}+3}}$
    \item $\displaystyle \lim_{x \to 3} {\frac{\left(x-3\right)^{2}}{x}}$
    \item $\displaystyle \lim_{x \to 1.5} {\left[x\right]}$\
    
    \textit{$\left[x\right]$ is the nearest integer function.}
    \item $\displaystyle \lim_{x \to \infty} {\frac{2^{x}}{x^{2}}}$
    \item $\displaystyle \lim_{x \to 0} {\left|\frac{1}{x^{3}}\right|}$
\end{enumerate}
\end{multicols}
\section{Piecewise Functions}
Functions that are defined differently for different values of the variable (\textbf{piecewise functions}) are often used to assess students' understanding of the characteristics of limits.

Consider the function:

\[
p(x) = \left\{
	\begin{array}{lr}
       x^{2} & \text{for } x \neq 2\\
       1 & \text{for } x = 2
     \end{array}
   \right.
\]

With the graph:

\begin{tikzpicture}
  \begin{axis}[ 
    xlabel=$x$,
    ylabel={$p(x)= x^2$},
    xmin=0, xmax=3,
    ymin=0, ymax=5,
  ] 
    \addplot[black,smooth] {(x^2)}; 
    \addplot[
    color=black,
    fill=white,
    mark=*,
    only marks
    ] coordinates{
    (2,4)
    };
    \addplot[
    color=black,
    fill=black,
    mark=*,
    only marks
    ] coordinates{
    (2,1)
    };
  \end{axis}
\end{tikzpicture}

It should be clear to the reader at this point that $\displaystyle\lim_{x \to 2} p(x) = 4$ regardless of the value assigned to $p(x)$ at $x=2$.

\chapter{Problem Set Solutions}
\newpage
\begin{multicols}{2}
\subsubsection*{Problem Set 1}
\begin{enumerate}
	\item Since $f(x)=x+4$ exists at $x=3$ and is continuous for all $x$, $\displaystyle \lim_{x \to 3} {x+4} = f(3) = 7$
    \item $\displaystyle \lim_{x \to 0} {\frac{1}{x}} = 0$
    \item Since the denominator grows considerably faster than the numerator, $\displaystyle \lim_{x \to \infty} {\frac{2-x}{x^{2}+3}} = 0$. It may be interesting to note that the function approaches 0 from the negative side of things.
    \item $\displaystyle \lim_{x \to 3} {\frac{\left(x-3\right)^{2}}{x}} = {\frac{\left(3-3\right)^{2}}{x}} = 0$
    \item Since $\displaystyle \lim_{x \to 1.5^{+}} {\left[x\right]} = 2 \neq \lim_{x \to 1.5^{-}} {\left[x\right]} = 1$, the limit does not exist.
    
    \item Since $2^{x}$ grows considerably faster than $x^{2}$, $\displaystyle \lim_{x \to \infty} {\frac{2^{x}}{x^{2}}} = \infty$
    \item $\displaystyle \lim_{x \to 0} {\left|\frac{1}{x^{3}}\right|} = \infty$
    
    Note that $\displaystyle \lim_{x \to 0} {\frac{1}{x^{3}}}$ does not exist.
\end{enumerate}
\end{multicols}
\end{document}